{\fontfamily{cmss}\selectfont
\scalefont{0.99}
\subsection{General}
\begin{itemize}[leftmargin=*, noitemsep]
  \item $|a - b| + |b - c| + |c - a| = 2 (\max(a, b, c) - \min(a, b, c))$

  \item $a \cdot b \leq c \to a \leq \left\lfloor \frac{c}{b} \right\rfloor$ is correct

  \item $a \cdot b < c \to a < \left\lfloor \frac{c}{b} \right\rfloor$ is incorrect

  \item $a \cdot b \geq c \to a \geq \left\lfloor \frac{c}{b} \right\rfloor$ is correct

  \item $a \cdot b > c \to a > \left\lfloor \frac{c}{b} \right\rfloor$ is correct

  \item For positive integer $n$, and arbitrary real numbers $m$, $x$:
  $$
  \left\lfloor \frac{\left\lfloor \frac{x}{m} \right\rfloor}{n} \right\rfloor = \left\lfloor \frac{x}{mn} \right\rfloor
  \text{and}
  \left\lceil \frac{\left\lfloor \frac{x}{m} \right\rfloor}{n} \right\rceil = \left\lceil \frac{x}{mn} \right\rceil
  $$
\end{itemize}

\subsection{Probabilty and Expected Value}
\begin{itemize} [leftmargin=*, noitemsep]
  \item \emph{Bayes Theorem:} $P(A \mid B) = \frac{P(A \cap B)}{P(B)} = \frac{P(B \mid A) \cdot P(A)}{P(B)}$
\end{itemize}

\subsection{Geometry}
\subsubsection{Triangles}
Circumradius: $R=\dfrac{abc}{4A}$ , Inradius: $r=\dfrac{A}{s}$\\
Length of median (divides triangle into two equal-area triangles): $m_a=\tfrac{1}{2}\sqrt{2b^2+2c^2-a^2}$\\
Length of bisector (divides angles in two): $s_a=\sqrt{bc\left[1-\left(\dfrac{a}{b+c}\right)^2\right]}$\\
Law of tangents: $\dfrac{a+b}{a-b}=\dfrac{\tan\dfrac{\alpha+\beta}{2}}{\tan\dfrac{\alpha-\beta}{2}}$\\
\subsubsection{Quadrilaterals}
With side lengths $a,b,c,d$, diagonals $e, f$, diagonals angle $\theta$, area $A$ and
magic flux $F=b^2+d^2-a^2-c^2$:

\[ 4A = 2ef \cdot \sin\theta = F\tan\theta = \sqrt{4e^2f^2-F^2} \]

 For cyclic quadrilaterals the sum of opposite angles is $180^\circ$,
$ef = ac + bd$, and $A = \sqrt{(p-a)(p-b)(p-c)(p-d)}$.

\subsubsection{Spherical coordinates}
\centerline{\includegraphics[width=25mm]{../code/sphericalCoordinates}}
\[\begin{array}{cc}
x = r\sin\theta\cos\phi & r = \sqrt{x^2+y^2+z^2}\\
y = r\sin\theta\sin\phi & \theta = \textrm{acos}(z/\sqrt{x^2+y^2+z^2})\\
z = r\cos\theta & \phi = \textrm{atan2}(y,x)
\end{array}\]

\subsubsection{3D figures} \\
\begin{tabular}{ll}
  Sphere & Volume $V=\frac{4}{3} \pi r^3$, surface area $S=4 \pi r^2$ \\
  Spherical sect. &
    Volume $V = \pi h^2 (r - h/3)$,
	surface area $S = 2 \pi r h$ \\
  Pyramid &
    Volume $V=\frac{1}{3} h S_{base}$ \\
  Cone &
    Volume $V=\frac{1}{3} \pi r^2 h$,
    lat. surf. area $S = \pi r \sqrt{r^2+h^2}$
\end{tabular}

\subsection{Binomial Coefficent}

\begin{itemize}[leftmargin=*, noitemsep]
  \item Factoring in: \( \binom{n}{k} = \frac{n}{k} \binom{n - 1}{k - 1} \)

  \item Sum over \( k \): \( \sum_{k = 0}^n \binom{n}{k} = 2^n \)

  \item Alternating sum: \( \sum_{k = 0}^n (-1)^k \binom{n}{k} = 0 \)

  \item Even and odd sum: \( \sum_{k = 0}^n \binom{n}{2k} = \sum_{k = 0}^n \binom{n}{2k + 1} 2^{n - 1} \)

  \item The Hockey Stick Identity

     - (Left to right) Sum over \( n \) and \( k \): \( \sum_{k = 0}^m \binom{n + k}{k} = \binom{n + m - 1}{m} \)

     - (Right to left) Sum over \( n \): \( \sum_{m = 0}^n \binom{m}{k} = \binom{n + 1}{k + 1} \)

  \item Sum of the squares: \( \sum_{k = 0}^n (\binom{n}{k})^2 = \binom{2n}{n} \)

  \item Weighted sum: \( \sum_{k = 1}^n k \binom{n}{k} = n2^{n - 1} \)

  \item Connection with the fibonacci numbers: \( \sum_{k = 0}\binom{n - k}{k} = F_{n + 1} \)

  \item Vandermonde's Identity: \( \sum_{i = 0}^k \binom{m}{i} \binom{n}{k - i} = \binom{m + n}{k} \)
  \item If \( f(n, k) = C(n, 0) + C(n, 1) + ... + C(n, k) \), Then \( f(n + 1, k) = 2 * f(n, k) - C(n, k) \) [For multiple \( f(n, k) \) queries, use Mo's algo]
\end{itemize}

\textbf{Lucas Theorem}

\[ \binom{m}{n} \equiv \prod_{i = 0}^k \binom{m_i}{n_i} (\mod p) \]

\begin{itemize}[leftmargin=*, noitemsep]
  \item \( \binom{m}{n} \) is divisible by p if and only if at least one of the base-\(p\) digits of \( n \) is greater than the corresponding base-\( p \) digit of \( m \).

  \item The number of entries in the \( n \)th row of Pascal's triangle that are not divisible by \( p  = \prod_{i = 0}^k (n_i + 1) \)

  \item All entries in the \( (p^{k} - 1)th \) row are not divisble by \( p \).
  \item \( \binom{n}{m} \equiv \lfloor \frac{n}{p} \rfloor (\mod p) \)
\end{itemize}

\subsection{Fibonacci Number}


% \begin{equation}
\textbf{1.} $k=A-B, F_A F_B=F_{k+1} F_A^2 + F_k F_A F_{A-1}$\\
\textbf{2.}$\sum_{i=0}^n F_i^2=F_{n+1} F_n$ \hspace{1cm}
\textbf{3.}$\sum_{i=0}^n F_i F_{i+1}=F_{n+1}^2-(-1)^n $\\
\textbf{4.}$\sum_{i=0}^n F_i F_{i+1}=F_{n+1}^2-(-1)^n$ \hspace{1cm}
\textbf{5.}$\sum_{i=0}^n F_i F_{i-1}=\sum_{i=0}^{n-1} F_i F_{i+1}$\\
\textbf{6.}$\operatorname{gcd}\left(F_m, F_n\right)=F_{\operatorname{gcd}(m, n)}$ \hspace{1cm}
\textbf{7.}$\sum_{0 \leq k \leq n}\left( \binom{n-k}{k} \right)=F_{n+1}$\\
\textbf{8.}$\operatorname{gcd}\left(F_n, F_{n+1}\right)=\operatorname{gcd}\left(F_n, F_{n+2}\right)= \operatorname{gcd}\left(F_{n+1}, F_{n+2}\right)=1$

\subsection{Sums}
  $1^4 + 2^4 + 3^4 + \dots + n^4 &= \frac{n(n+1)(2n+1)(3n^2 + 3n - 1)}{30}$ \\
  $\Scale[0.85]{ \sum\limits_{i=1}^{n} i^{m} = \frac{1}{m + 1}  \left[ (n + 1)^{m + 1} - 1 - \sum\limits_{i=1}^{n} \left((i+1)^{m+1} - i^{m+1} - (m + 1)i^{m}\right)\right]} $\\
  $\Scale[0.99]{\sum\limits_{i=1}^{n-1} i^{m} = \frac{1}{m + 1} \sum\limits_{k=0}^{m} { {m+1}\choose{k} } B_{k}n^{m + 1 - k}}$\\
  $\sum\limits_{k=0}^n kx^k = (x - (n+1)x^{n+1} + nx^{n+2})/(x-1)^2$\\
  $\sum_{i = 0}^{n} i \times i! = (n + 1)! - 1$

\subsection{Polynomials and Series}
\begin{itemize}[leftmargin=*, noitemsep]
  \item $x^n - y^n = (x - y) (\sum_{i=0}{n-1} x^{n - i - 1}y^{i})$
  \item $\sum_{i=1,3,5,\dots}^{i \leq n} \binom{n}{i} a^{n-i} b^i = \frac{1}{2} \left( (a+b)^n - (a-b)^n \right)$
  \item $e^x = 1+x+\frac{x^2}{2!}+\frac{x^3}{3!}+\dots,\,(-\infty<x<\infty)$
  \item $\ln(1+x) = x-\frac{x^2}{2}+\frac{x^3}{3}-\frac{x^4}{4}+\dots,\,(-1<x\leq1)$
  \item $\sqrt{1+x} = 1+\frac{x}{2}-\frac{x^2}{8}+\frac{2x^3}{32}-\frac{5x^4}{128}+\dots,\,(-1\leq x\leq1)$
  \item $\sin x = x-\frac{x^3}{3!}+\frac{x^5}{5!}-\frac{x^7}{7!}+\dots,\,(-\infty<x<\infty)$
  \item $\cos x = 1-\frac{x^2}{2!}+\frac{x^4}{4!}-\frac{x^6}{6!}+\dots,\,(-\infty<x<\infty)$
  \item $(x + a)^{-n} = \sum\limits_{k=0}^{\infty} (-1)^{k} { {n + k - 1}\choose{k}} x^{k}a^{-n-k}$
  \item $ 1/(1 - x) = 1 + x + x^2 + x^3 + ... $
  \item $ 1/(1 - ax) = 1 + ax + (ax)^2 + (ax)^3 + ... $
  \item $ 1/(1 - x)^2 = 1 + 2x + 3x^2 + 4x^3 + ... $
  \item $ 1/(1 - x)^3 = C(2, 2) + C(3, 2)x + C(4, 2)x^2 + C(5, 2)x^3 + ... $
  \item $ 1/(1 - ax)^{k + 1} = 1 + C(1 + k, k)(ax) + C(2 + k, k)(ax)^2 + C(3 + k, k)(ax)^3 + ... $
  \item $ x(x + 1)(1 - x)^{-3} = 1 + x + 4x^2 + 9x^3 + 16x^4 + 25x^5 + ... $
\end{itemize}

\subsection{Pythagorean Triples}
 The Pythagorean triples are uniquely generated by
 \[ a=k\cdot (m^{2}-n^{2}),\ \,b=k\cdot (2mn),\ \,c=k\cdot (m^{2}+n^{2}), \]
 with $m > n > 0$, $k > 0$, $m \bot n$, and either $m$ or $n$ even.


\subsection{Number Theory}

\begin{itemize}[leftmargin=*, noitemsep]
\item HCN: 1e6(240), 1e9(1344), 1e12(6720), 1e14(17280), 1e15(26880), 1e16(41472)
\item \( gcd(a, b, c, d, ...) = gcd(a, b - a, c - b, d - c, ...) \)
\item \( gcd(a + k, b + k, c + k, d + k, ...) = gcd(a + k, b - a, c - b, d - c, ...) \)
\item Primitive root exists iff \( n = 1, 2, 4, p^k, 2\times p^k \), where \( p \) is an odd prime.
\item If primtive root exists, there are \( \phi(\phi(n)) \) primtive roots of \( n \).
\item The numbers from \( 1 \) to \( n \) have in total \( O(n\log\log n) \) unique prime factors.
\item \( x \equiv r_1 \mod m1 \) and \( x \equiv r_2 \mod m2 \) has a solution iff \( \gcd(m_1, m_2) | (r_1 - r_2) \)
Solution of \( x^2 \equiv a (\mod p) \)
\item \( ca \equiv cb \pmod{m} \iff a \equiv b \pmod{ \frac{n}{\gcd(n, c)}} \)
\item \( ax \equiv b \pmod{m} \) has a solution \( \iff \) \( \gcd(a, m) | b \)
\item If \( ax \equiv b \pmod{m} \) has a solution, then it has \( gcd(a, m) \) solutions and they are separated by \( \frac{m}{\gcd(a, m)} \)
\item \( ax \equiv 1 \pmod{m} \) has a solution or \( a \) is invertible \( \pmod{m} \) \( \iff \) \(\gcd(a, m) = 1 \)
\item \( x^2 \equiv 1 \pmod{p} \) then \( x \equiv \pm 1 \pmod{p} \)
\item There are \( \frac{p - 1}{2} \) has no solution.
\item There are \( \frac{p - 1}{2} \) has exaclty two solutions.
\item When \( p \% 4 = 3 \), \( x \equiv \pm a^{\frac{p + 1}{4}} \)
\item When \( p \% 8 = 5 \), \( x \equiv a^{\frac{p + 3}{8}} \; or \; x \equiv 2^{\frac{p - 1}{4}} a^{\frac{p + 3}{8}} \)
\end{itemize}

\subsubsection{Primes}
  $p=962592769$ is such that $2^{21} \mid p-1$, which may be useful. For hashing
  use 970592641 (31-bit number), 31443539979727 (45-bit), 3006703054056749
  (52-bit). There are 78498 primes less than 1\,000\,000.

  Primitive roots exist modulo any prime power $p^a$, except for $p = 2, a > 2$, and there are $\phi(\phi(p^a))$ many.
  For $p = 2, a > 2$, the group $\mathbb Z_{2^a}^\times$ is instead isomorphic to $\mathbb Z_2 \times \mathbb Z_{2^{a-2}}$.

\subsubsection{Estimates}
  $\sum_{d|n} d = O(n \log \log n)$.

  The number of divisors of $n$ is at most around 100 for $n < 5e4$, 500 for $n < 1e7$, 2000 for $n < 1e10$, 200\,000 for $n < 1e19$.

\subsubsection{Perfect numbers}  $n>1$ is called perfect if it equals
sum of its proper divisors and $1$.  Even $n$ is perfect iff $n = 2^{p-1} (2^p - 1)$
and $2^p - 1$ is prime (Mersenne's). No odd perfect numbers are yet found.

\subsubsection{Carmichael numbers}
A positive composite $n$ is a Carmichael number
($a^{n-1} \equiv 1 \pmod{n}$ for all $a$: $\gcd(a,n)=1$),
iff $n$ is square-free, and for all prime divisors $p$ of $n$, $p-1$ divides $n-1$.

\subsubsection{Totient}

\begin{itemize}[leftmargin=*, noitemsep]
  \item If $ p $ is a prime $ \phi(p^k) = p^k - p^{k-1} = p^{k}(1 - \frac{1}{p})$
  \item If $ a $ and $ b $ are relatively prime, $ \phi(ab) = \phi(a)\phi(b) $
  \item $ \phi(n) = n(1-\frac{1}{p_1})(1-\frac{1}{p_2})(1-\frac{1}{p_3})...(1-\frac{1}{p_k}) $
  \item Sum of coprime to $ n = n * \frac{\phi(n)}{2} $
  \item If $ n = 2^k, \phi(n) = 2^{k - 1} = \frac{n}{2} $
  \item For $ a $ and $ b $, $ \phi(ab) = \phi(a)\phi(b)\frac{d}{\phi(d)} $
  \item $ \phi (ip) = p \phi(i) $ whenever $ p $ is a prime and it divides $ i $
  \item The number of $ a (1<= a <=N) $ such that $ gcd(a, N)=d $ is $ \phi(\frac{n}{d}) $
  \item If $ n > 2 $ , $ \phi(n) $ is always even
  \item Sum of gcd, $ \sum_{i=1}^n gcd(i, n) = \sum_{d|n} d \phi(\frac{n}{d}) $
  \item Sum of lcm, $ \sum_{i=1}{n}lcm(i, n) = \frac{n}{2}(\sum_{d|n}(d \phi(d))+1) $
  \item $ \phi(1) = 1 $ and $ \phi(2) = 1 $ which two are only odd $ \phi $
  \item $ \phi(3) = 2 $ and $ \phi(4) = 2 $ and $ \phi(6) = 2 $ which three are only prime $ \phi $
  \item Find minimum n such that $ \frac{\phi(n)} {n} $ is  maximum- Multiple of small primes- $ 2 * 3 * 5 * 7 * 11 * 13 * ... $
\end{itemize}

\subsubsection{Mobius function}
\begin{itemize}[leftmargin=*, noitemsep]
  \item $\sum_{d \mid n} \phi(d) = n$
  \item $\sum_{d \mid n} \mu(d) = [n = 1]$
  \item $\phi(n) = \sum_{d|n} \mu(d) \frac{n}{d}$.
  \item $\text{\emph{Number of coprime tuples}} = \sum_{i} \mu(i) \cdot \text{cnt}_i$
  \item $\text{\emph{Sum of gcd of all tuples}} = \sum_{i} \phi(i) \cdot \text{cnt}_i$
  \item $\text{\emph{Sum of lcm of all tuples}} = \sum_{i} f(i) \cdot \mu(i) \cdot \text{cnt}_i$
  \begin{itemize}[leftmargin=0.5cm, noitemsep]
    \item $f(i) = \frac{1}{i} \sum_{d \mid i} d \cdot \mu(d)$
  \end{itemize}
  \item If for all $n \in N$, $F(n)=\sum_{d|n} f(d)$, then $f(n) = \sum_{d|n} \mu(d) F(\frac{n}{d})$, and vice versa.
  \item If $f$ is multiplicative, then $\sum_{d|n} \mu(d) f(d) = \prod_{p|n}(1-f(p))$,
  $\sum_{d|n} \mu(d)^2 f(d) = \prod_{p|n} (1+f(p))$.
\end{itemize}

\subsubsection{Legendre symbol} If $p$ is an odd prime, $a \in {\mathbb Z}$, then
$\left(\frac{a}{p}\right)$ equals $0$, if $p | a$; $1$ if $a$ is a quadratic
residue modulo $p$; and $-1$ otherwise.
Euler's criterion:
$\left(\frac{a}{p}\right)=a^{\left(\frac{p-1}{2}\right)} \pmod p$. \\
%$\left(\frac{a}{p}\right) \left(\frac{b}{p}\right) = \left(\frac{ab}{p}\right)$
%Law of Quadratic Reciprocity: for any distinct odd primes $p$ and $q$,
%$\left(\frac{p}{q}\right) \left(\frac{q}{p}\right) = (-1)^{\frac{p-1}{2} \cdot \frac{q-1}{2}}$
\subsubsection{Jacobi symbol}  %Generalization of Legendre's symbol to any odd modulus. \\
If $n=p_1^{a_1} \cdots p_k^{a_k}$ is odd, then
$\left(\frac{a}{n}\right) = \prod_{i=1}^k \left(\frac{a}{p_i}\right)^{k_i}$.

%\subsubsection{Kronecker symbol}
%Let $a$ be a positive integer, which is not a perfect square and
%$a \equiv 0$ or $1 {\pmod 4}$. \\
%$\left(\frac{a}{2}\right) = \{ 1$, if $a \equiv 1 {\pmod 8}$;
%$-1$, if $a \equiv 5 {\pmod 8} \}$. \\
%$\left(\frac{a}{n}\right) = \prod_{j=1}^k p_j^{k_j}$,
%if gcd$(a,n) \ne 1$ and $n=\prod p_i^{k_i}$.
%$\left(\frac{a}{n}\right)$ equals Jacobi symbol otherwise.

\subsubsection{Primitive roots}  If the order of $g$ modulo $m$ (min $n>0$:
$g^n \equiv 1 \pmod{m}$) is $\phi(m)$, then $g$ is called a primitive root.
If $Z_m$ has a primitive root, then it has $\phi(\phi(m))$ distinct primitive
roots. $Z_m$ has a primitive root iff $m$ is one of $2$, $4$,
$p^k$, $2p^k$, where $p$ is an odd prime.
If $Z_m$ has a primitive root $g$, then for all $a$ coprime to $m$,
there exists unique integer $i=\text{ind}_g(a)$ modulo $\phi(m)$,
such that $g^i \equiv a \pmod{m}$.
$\text{ind}_g(a)$ has logarithm-like properties:
$\text{ind}(1) = 0$, $\text{ind}(ab) = \text{ind}(a) + \text{ind}(b)$.

If $p$ is prime and $a$ is not divisible by $p$, then congruence
$x^n \equiv a \pmod{p}$ has $\gcd(n, p-1)$ solutions if
$a^{(p-1)/\gcd(n,p-1)} \equiv 1 \pmod{p}$, and no solutions otherwise.
(Proof sketch: let $g$ be a primitive root, and
$g^i \equiv a \pmod{p}$, $g^u \equiv x \pmod{p}$.
$x^n \equiv a \pmod{p}$ iff $g^{nu} \equiv g^i \pmod{p}$ iff $nu \equiv i \pmod{p}$.)

\subsubsection{Discrete logarithm problem}  Find $x$ from $a^x \equiv b \pmod{m}$.
Can be solved in $O(\sqrt{m})$ time and space with a meet-in-the-middle trick.
Let $n = \lceil \sqrt{m} \rceil$, and $x = ny - z$.
Equation becomes $a^{ny} \equiv b a^z \pmod{m}$.  Precompute all values that
the RHS can take for $z = 0, 1, \dots, n-1$, and brute force $y$ on the LHS,
each time checking whether there's a corresponding value for RHS.

\subsubsection{Postage stamps/McNuggets problem}  Let $a$, $b$ be relatively-prime integers.
There are exactly $\frac{1}{2}(a-1)(b-1)$ numbers \emph{not} of form $ax+by$ ($x,y \ge 0$),
and the largest is $(a-1)(b-1)-1 = ab - a - b$.

\subsubsection{Fermat's two-squares theorem}  Odd prime $p$ can be represented
as a sum of two squares iff $p \equiv 1 {\pmod 4}$.
A product of two sums of two squares is a sum of two squares.
Thus, $n$ is a sum of two squares iff every prime of
form $p=4k+3$ occurs an even number of times in $n$'s factorization.

% }}}

\subsection{Twelve-fold Way}
\begin{center}
\begin{tabular}{c|c|c|c|c}
% \hline
\textbf{Balls} & \textbf{Urns} & \textbf{unrestricted} & $\leq 1$ & $\geq 1$ \\ 
\hline
labeled & labeled & $u^b$ & $(u)_b$ & $u! S(b, u)$ \\ 
% \hline
unlabeled & labeled & $\binom{u + b - 1}{b}$ & $\binom{u}{b}$ & $\binom{b-1}{u-1}$ \\ 
% \hline
labeled & unlabeled & $\sum_{i=1}^u S(b, i)$ & $[b \leq u]$ & $S(b, u)$ \\ 
% \hline
unlabeled & unlabeled & $\sum_{i=1}^u p_i(b)$ & $[b \leq u]$ & $p_u(b)$ \\ 
% \hline
\end{tabular}
\end{center}

\subsection{Permutations}
  \subsubsection{Cycles}
    Let $g_S(n)$ be the number of $n$-permutations whose cycle lengths all belong to the set $S$. Then
    $$\sum_{n=0} ^\infty g_S(n) \frac{x^n}{n!} = \exp\left(\sum_{n\in S} \frac{x^n} {n} \right)$$

  \subsubsection{Derangements}
    Permutations of a set such that none of the elements appear in their original position.
    \[ \mkern-2mu D(n) = (n-1)(D(n-1)+D(n-2)) = n D(n-1)+(-1)^n = \left\lfloor\frac{n!}{e}\right\rceil \]

  \subsubsection{Involutions}
  An involution is a permutation with maximum cycle length 2, and it is its own inverse.

  \[
  a(n) = a(n - 1) + (n - 1)a(n - 2)
  \]
  \[
  a(0) = a(1) = 1
  \]
  
  \[
  1, 1, 2, 4, 10, 26, 76, 232, 764, 2620, 9496, 35696, 140152
  \]
  \subsubsection{Burnside's lemma}
    Given a group $G$ of symmetries and a set $X$, the number of elements of $X$ \emph{up to symmetry} equals
     \[ {\frac {1}{|G|}}\sum _{{g\in G}}|X^{g}|, \]
     where $X^{g}$ are the elements fixed by $g$ ($g.x = x$).

     If $f(n)$ counts ``configurations'' (of some sort) of length $n$, we can ignore rotational symmetry using $G = \mathbb Z_n$ to get
     \[ g(n) = \frac 1 n \sum_{k=0}^{n-1}{f(\text{gcd}(n, k))} = \frac 1 n \sum_{k|n}{f(k)\phi(n/k)} \]

\subsection{Partitions and subsets}
  \subsubsection{Partition function}
    Number of ways of writing $n$ as a sum of positive integers, disregarding the order of the summands.
    \[ p(0) = 1,\ p(n) = \sum_{k \in \mathbb Z \setminus \{0\}}{(-1)^{k+1} p(n - k(3k-1) / 2)} \]
    \[ p(n) \sim 0.145 / n \cdot \exp(2.56 \sqrt{n}) \]

    \begin{center}
    \begin{tabular}{c|c@{\ }c@{\ }c@{\ }c@{\ }c@{\ }c@{\ }c@{\ }c@{\ }c@{\ }c@{\ }c@{\ }c@{\ }c}
      $n$    & 0 & 1 & 2 & 3 & 4 & 5 & 6  & 7  & 8  & 9  & 20  & 50  & 100 \\ \hline
      $p(n)$ & 1 & 1 & 2 & 3 & 5 & 7 & 11 & 15 & 22 & 30 & 627 & $\mathtt{\sim}$2e5 & $\mathtt{\sim}$2e8 \\
    \end{tabular}
    \end{center}

  \subsubsection{Partition Number}

- Time Complexity: $ O(n\sqrt{n}) $
\begin{verbatim}
for (int i = 1; i <= n; ++i) {
  pent[2 * i - 1] = i * (3 * i - 1) / 2;
  pent[2 * i] = i * (3 * i + 1) / 2;
}
p[0] = 1;
for (int i = 1; i <= n; ++i) {
  p[i] = 0;
  for (int j = 1, k = 0; pent[j] <= i; ++j) {
    if (k < 2) p[i] = add(p[i], p[i - pent[j]]);
    else p[i] = sub(p[i], p[i - pent[j]]); ++k, k &= 3;
  }
}
\end{verbatim}
- The number of partitions of a positive integer \( n \) into exactly \( k \) parts equals the number of partitions of \( n \) whose largest part equals \( k \)

\[ p_k(n) = p_k(n - k) + p_{k - 1}(n - 1) \]

\subsubsection{2nd Kaplansky's Lemma}

The number of ways of selecting \( k \) objects, no two consecutive,from \( n \) labelled objects arrayed in a circle is \( \frac{n}{k} \binom{n-k-1}{k-1} = \frac{n}{n - k} \binom{n-k}{k} \)

\subsubsection{Distinct Objects into Distinct Bins}

- $ n $ distinct objects into $ r $ distinct bins $ = r^n $

- Among $ n $ distinct objects, exactly $ k $ of them into r distincts bins $ = \binom{n}{k}r^k $

- $ n $ distinct objects into $ r $ distinct bins such that each bin contains at least one object $ = \sum_{i = 0}^{r} (-1)^i \binom{r}{i} (r - i)^n $

\subsection{Coloring}

\begin{itemize}[leftmargin=*, noitemsep]
  \item The number of labeled undirected graphs with \( n \) vertices, \( G_n = 2^{\binom{n}{2}} \)

  \item The number of labeled directed graphs with \( n \) vertices, \( G_n = 2^{n(n \item 1)} \)

  \item The number of connected labeled undirected graphs with \( n \) vertices, \( C_n = 2^{\binom{n}{2}} - \frac{1}{n} \sum_{k = 1}^{n - 1} k \binom{n}{k} 2^{\binom{n-k}{2}}C_k = 2^{\binom{n}{2}} - \sum_{k = 1}^{n - 1} \binom{n - 1}{k - 1} 2^{\binom{n-k}{2}}C_k \)

  \item The number of k-connected labeled undirected graphs with \( n \) vertices, \( D[n][k] = \sum_{s = 1}^{n} \binom{n - 1}{s- 1}C_s D[n - s][k - 1] \)

  \item Cayley's formula: the number of trees on \( n \) labeled vertices = the number of spanning trees of a complete graph with \( n \) labeled vertices = \( n^{n - 2} \)

  \item Number of ways to color a graph using k color such that no two adjacent nodes have same color

    \begin{itemize}[leftmargin=*, noitemsep]
      \item Complete graph = \( k(k-1)(k-2)...(k-n+1) \)
      \item Tree = \( k(k - 1)^{n - 1} \)
      \item Cycle = \( (k - 1)^n + (-1)^n (k - 1) \)
    \end{itemize}

  \item Number of trees with $ n $ labeled nodes: $ n^{n - 2} $
\end{itemize}

\subsection{General purpose numbers}
  \subsubsection{Eulerian numbers}a
    Number of permutations $\pi \in S_n$ in which exactly $k$ elements are greater than the previous element. $k$ $j$:s s.t. $\pi(j)>\pi(j+1)$, $k+1$ $j$:s s.t. $\pi(j)\geq j$, $k$ $j$:s s.t. $\pi(j)>j$.
    $$E(n,k) = (n-k)E(n-1,k-1) + (k+1)E(n-1,k)$$
    $$E(n,0) = E(n,n-1) = 1$$
    $$E(n,k) = \sum_{j=0}^k(-1)^j\binom{n+1}{j}(k+1-j)^n$$

  % \subsubsection{Stirling numbers of the first kind}
    


  % \subsubsection{Stirling numbers of the second kind}
  %   Partitions of $n$ distinct elements into exactly $k$ groups.
  %   $$S(n,k) = S(n-1,k-1) + k S(n-1,k)$$
  %   $$
  %   $$

    % - Time Complexity: $ O(k\log n) $
  % \begin{verbatim}
  
  % \end{verbatim}

  \subsubsection{Bell numbers}
    Total number of partitions of $n$ distinct elements. $B(n) =$
    $1, 1, 2, 5, 15, 52, 203, 877, 4140, 21147, \dots$. For $p$ prime,
    \[ B(p^m+n)\equiv mB(n)+B(n+1) \pmod{p} \]

  \subsubsection{Bernoulli numbers}
  $\Scale[.95]{\sum\limits_{j=0}^m {m+1 \choose j} B_j = 0$.
  \quad $B_0=1$, $B_1=-\frac{1}{2}$. $B_n=0$, for all odd $n \ne 1}$.

  \subsubsection{Catalan numbers}
    \[ C_n=\frac{1}{n+1}\binom{2n}{n}= \binom{2n}{n}-\binom{2n}{n+1} = \frac{(2n)!}{(n+1)!n!} \]
    \[ C_0=1,\ C_{n+1} = \frac{2(2n+1)}{n+2}C_n,\ C_{n+1}=\sum C_iC_{n-i} \]
    \begin{itemize}[leftmargin=*, noitemsep]
      \item ${C_n = 1, 1, 2, 5, 14, 42, 132, 429, 1430, 4862, 16796, 58786, \dots}$
      \item sub-diagonal monotone paths in an $n\times n$ grid.
      \item strings with $n$ pairs of parenthesis, correctly nested.
      \item binary trees with with $n+1$ leaves (0 or 2 children).
      \item ordered trees with $n+1$ vertices.
      \item ways a convex polygon with $n+2$ sides can be cut into triangles by connecting vertices with straight lines.
      \item permutations of $[n]$ with no 3-term increasing subseq.
      \item \textbf{Non-Crossing Partitions:} Ways to partition a set of $n$ elements such that no two blocks of the partition "cross" when visualized on a circle.
      \item Find the count of balanced parentheses sequences consisting of $n + k$ pairs of parentheses where the first $k$ symbols are open brackets.
   $$C_n^{(k)} = \frac{k + 1}{n + k + 1} \binom{2n + k}{n}$$
      \item Recursive formula of Catalan Numbers:
    $$ C_{n}^{(k)} = \frac{(2n + k - 1) \cdot (2n + k)}{n \cdot (n + k + 1)} C_{n - 1}^{(k)} $$
    \end{itemize}
  
    \subsubsection{Super Catalan numbers}
    The number of monotonic lattice paths of a \( n \times n \)-grid that do not touch the diagonal.
    
    \[
    S(n) = \frac{3(2n - 3)S(n - 1) - (n - 3)S(n - 2)}{n}
    \]
    \[
    S(1) = S(2) = 1
    \]
    
    \[
    1, 1, 3, 11, 45, 197, 903, 4279, 20793, 103049, 518859
    \]
    
    \subsubsection{Motzkin numbers}
    Number of ways of drawing any number of nonintersecting chords among \( n \) points on a circle. Number of lattice paths from \((0, 0)\) to \((n, 0)\) never going below the \( x \)-axis, using only steps NE, E, SE.
    
    \[
    M(n) = \frac{3(n - 1)M(n - 2) + (2n + 1)M(n - 1)}{n + 2}
    \]
    \[
    M(0) = M(1) = 1
    \]
    
    \[
    1, 1, 2, 4, 9, 21, 51, 127, 323, 835, 2188, 5798, 15511, 41835, 113634
    \]

  \subsection{Narayana numbers}
  \[ N(n, k) = \frac{1}{n} \binom{n}{k} \binom{n}{k - 1} \]
  \begin{itemize}[leftmargin=*, noitemsep]
    \item $n$ pairs of balanced parenthesis with $k$ nestings.
    \item number of lattice paths from $(0, 0)$ to $(2n,0)$, with steps only northeast and southeast and k peaks.
    \item number of unlabeled ordered rooted trees with $n$ edges and $k$ leaves.
    \item number of non-crossing partition of a set with $n$ elements into exactly $k$ blocks.
  \end{itemize}

  \subsubsection{Schröder numbers}
Number of lattice paths from \((0, 0)\) to \((n, n)\) using only steps \(N, NE, E\), never going above the diagonal. Number of lattice paths from \((0, 0)\) to \((2n, 0)\) using only steps \(NE, SE\) and double east \(EE\), never going below the \(x\)-axis. Twice the Super Catalan number, except for the first term.

\[
1, 2, 6, 22, 90, 394, 1806, 8558, 41586, 206098
\]


  \subsubsection{Lucas Number}

  Number of edge cover of a cycle graph $ C_n $ is $ L_n $

$$ L(n) = L(n-1) + L(n-2); L(0)=2, L(1)=1 $$

\subsection{Ballot Theorem}

Suppose that in an election, candidate A receives
a votes and candidate B receives b votes, where a ≥ kb for some positive
integer k. Compute the number of ways the ballots can be ordered so that
A maintains more than k times as many votes as B throughout the counting
of the ballots.

The solution to the ballot problem is $\frac{a - kb}{a+b} \times C(a + b, a)$

\subsection{Classical Problems}

\begin{itemize}[leftmargin=*, noitemsep]
  \item $ F(n, k) = $ number of ways to color n objects using exactly $ k $ colors. Let $ G(n, k) $ be the number of ways to color n objects using no more than $ k $ colors. Then, $ F(n, k) = G(n, k) - C(k, 1) * G(n, k-1) + C(k, 2) * G(n, k-2) - C(k, 3) * G(n, k-3) ... $
  \item Number of ways to divide $n$ persons into $\frac{n}{k}$ equal groups, each having size $k$ is:
  $\frac{n!}{k!^{\frac{n}{k}} \left( \frac{n}{k} \right)!} = \prod_{n \geq k}^{n -= k} \binom{n-1}{k-1}$
  \item Number of ways to choose $n$ ids from  $1$ to $b$ such that every id has distance at least $k$ is $\binom{b - (n-1)(k-1)}{n}$
\end{itemize}

\textbf{Determining G(n, k)} :

Suppose, we are given a 1 * n grid. Any two adjacent cells can not have same color.
Then, $G(n, k) = k * ((k-1)^(n-1))$

If no such condition on adjacent cells.
Then, $G(n, k) = k ^ n$

\subsection{Matching Formula}

\subsubsection{Normal Graph} \\
MM + MEC = n (exculding vertex), IS + VC = G, MIS + MVC = G \\

\subsubsection{Bipartite Graph} \\
MIS = n - MBM, MVC = MBM, MEC = n - MBM \\

\subsubsection{Grundy numbers}
For a two-player, normal-play (last to move wins) game on a graph $(V,E)$:
$G(x) = \mbox{mex}(\{ G(y) : (x, y) \in E \})$. $x$ is losing iff $G(x) = 0$.

\subsubsection{Sums of games}
\begin{itemize}[leftmargin=*, noitemsep]
  \item
    \emph{Player chooses a game and makes a move in it.}
    Grundy number of a position is xor of grundy numbers of positions in summed games.
  \item
    \emph{Player chooses a non-empty subset of games (possibly, all) and makes moves in all of them.}
    A position is losing iff each game is in a losing position.
  \item
    \emph{Player chooses a proper subset of games (not empty and not all),
        and makes moves in all chosen ones.}
    A position is losing iff grundy numbers of all games are equal. 
  \item
    \emph{Player must move in all games, and loses if can't move in some game}
    A position is losing if any of the games is in a losing position.
\end{itemize}

\subsubsection{Mis\`{e}re Nim}
A position with pile sizes $a_1, a_2, \dots, a_n \ge 1$,
not all equal to $1$, is losing iff $a_1 \oplus a_2 \oplus \dots \oplus a_n = 0$
(like in normal nim.)
A position with $n$ piles of size $1$ is losing iff $n$ is \emph{odd}.

\subsection{Tree Hashing}

\( f(u) = sz[u] * \sum_{i = 0} f(v) * p^{i} \);    \( f(v) \) are sorted \)
\( f(child) = 1 \)

\subsection{Permutation} \\
To maximize the sum of adjacent differences of a permuation, it is necessary and sufficient to place the smallest half numbers in odd position and the greatest half numbers in even position. Or, vice versa.

\subsection{String} \\

\begin{itemize}[leftmargin=*, noitemsep]
  \item If the sum of length of some strings is \( N \), there can be at most \( \sqrt{N} \) distinct length.
  \item A Text can have at most \( O(N \times \sqrt{N}) \) distinct substrings that match with given patterns where the sum of the length of the given patterns is \( N \).
  \item Period =  n \% (n - pi.back() == 0)? n - pi.back(): n
  \item The first \( (period) \) cyclic rotations of a string are distinct. Further cyclic rotations repeat the previous strings.
  \item \( S \) is a palindrome if and only if it's period is a palindrome.
  \item If $S$ and $T$ are palindromes, then the periods of $S$ \text{and} $T$ are same if and only if $S + T$ is a palindrome.
\end{itemize}

\subsection{Bit}
\begin{itemize}[leftmargin=*, noitemsep]
  \item (a xor b) and (a + b) has the same parity
  \item (a + b) = (a xor b) + 2 (a \& b) = (a | b) + (a \& b)\\
  \item gcd(a, b) <= a - b <= xor(a, b)
\end{itemize}

\subsection{Convolution}
\begin{itemize}[leftmargin=*, noitemsep]
  \item Hamming Distance: Replace \( 0 \) with \( -1 \)
  \item SQRT Decomposition: Find block size, B = sqrt(8 * n)
  \item Pattern Matching: 
  \begin{enumerate}[leftmargin=0.2cm, noitemsep]
    \item $A(x) = a_0x^0 + a_1x^1 + \cdots + a_{n-1}x^{n-1}, \quad n = |T|, a_i = \cos(\alpha_i) + i \sin(\alpha_i), \quad \alpha_i = \frac{2\pi T[i]}{26}.$
    \item $B(x) = b_0x^0 + b_1x^1 + \cdots + b_{m-1}x^{m-1}, \quad m = |P|, b_i = \cos(\beta_i) - i \sin(\beta_i), \quad \beta_i = \frac{2\pi P[m - i - 1]}{26}.$
    \item $C(x) = A(x) \times B(x)$, if $c_{m - 1 + i} = m$ then pattern appears in the text at position $i$.
  \end{enumerate}
  \item Pattern Matching with Wildcards: set $b_i = 0$ if $P[m-i-1] = *$. If $x$ is the number of wildcards in $P$, then we will have a match of $P$ in $T$ at index $i$ if $c_{m-1+i} = m - x$.
\end{itemize}
}